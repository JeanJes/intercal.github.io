\documentclass[a4paper, 11pt]{article}
\usepackage{comment} % enables the use of multi-line comments (\ifx \fi) 
\usepackage{lipsum} %This package just generates Lorem Ipsum filler text. 
\usepackage{fullpage} % changes the margin
\usepackage[a4paper, total={7in, 10in}]{geometry}
\usepackage[fleqn]{amsmath}
\usepackage{amssymb,amsthm}  % assumes amsmath package installed
\newtheorem{theorem}{Theorem}
\newtheorem{corollary}{Corollary}
\usepackage{graphicx}
\usepackage{tikz}
\usetikzlibrary{arrows}
\usepackage{verbatim}
\usepackage{amsmath} 
\usepackage{float}
\usepackage{xcolor}
\usepackage{mdframed}
\usepackage[shortlabels]{enumitem}
\usepackage{indentfirst}
\usepackage{hyperref}
\usepackage{dsfont}

\begin{document}
PRODUCTO PUNTO\\
Para el estudio de la topologia en $\mathds{R}^n$ tenemos que estudiar una operaci\'on dentro de nustro espacio vectorial que es de mucha utilidad para definir una norma y con ello una m\'etrica que llamamos Producto punto y tiene la siguiente

\(\textit{Definici\'on}:\)

El \texttt{Producto punto} es una funci\'on $<;>$ que en general $<;> : \mathds{R}^n \rightarrow \mathds{R}$ tal que si $\vec{x} = (x_1, x_2, \cdots, x_n)$ y $\vec{y} = (y_1, y_2, \cdots, y_n)$ entonces 
$$
<\vec{x} ; \vec{y}> = \vec{x} \cdot \vec{y} = \sum_{i=1}^n{x_i y_i} 
$$
De donde salen cuatro propiedades:
Consideremos los vectores $\vec{v},\vec{u}, \vec{w} \in \mathds{R}^n$ $\&$ $\alpha \in \mathds{R}$ entonces

\begin{enumerate}
\item $\vec{v} \cdot \vec{0} = 0$
\item $\vec{v} \cdot \vec{w} = \vec{w} \cdot \vec{v}$
\item $\vec{u} \cdot (\vec{v} + \vec{w}) = \vec{u} \cdot \vec{v} + \vec{u} \cdot \vec{w}$
\item $(\alpha \vec{v}) \cdot \vec{w} = \alpha (\vec{v} \cdot \vec{w})$
\end{enumerate}
\textit{Dem. de 1:}
\begin{center}
Sean $\vec{v},\vec{0} \in \mathds{R}^n$ $\&$ $\alpha \in \mathds{R}$ entonces, 
$\vec{0} = (0,0, \cdots, 0)$, n-veces, es decir un vector con todas sus entradas iguales a cero.
\end{center}

$$
    \left.
    \begin{array}{cclc}
     \vec{v} \cdot \vec{0} & = & (v_1, v_2, \cdots, v_n) \cdot (0,0, \cdots, 0) & 
  \\  & = & (v_1)0 + (v_2)0 + (v_3)0 + \cdots + (v_n)0 & Def. de <;>
  \\ & = & 0 + 0 + 0 + \cdots + 0  & $Propiedades en $\mathds{R}
  \\ & = & 0
     \end{array}
     \right.     
$$
$$ \therefore \vec{v} \cdot \vec{0} = 0     $$
\begin{flushright}
$\blacksquare$
\end{flushright}
y continuaremos con los siguientes incisos.\\
\textit{Dem. de 2:}
\begin{center}
Sean $\vec{v}, \vec{w} \in \mathds{R}^n$
\end{center}
$$
    \left.
    \begin{array}{cclc}
     \vec{v} \cdot \vec{w} & = & (v_1, v_2, \cdots, v_n) \cdot (w_1, w_2, \cdots, w_n) & 
  \\  & = & (v_1)w_1 + (v_2)w_2 + (v_3)w_3 + \cdots + (v_n)w_n & Def. de <;>
  \\  & = & w_1(v_1) + w_2(v_2) + w_3(v_3) + \cdots +  w_n(v_n) & $Propiedades en $\mathds{R}
  \\  & = & (w_1, w_2, \cdots, w_n) \cdot (v_1, v_2, \cdots, v_n) & Def. de <;>
  \\  & = & \vec{w} \cdot \vec{v}
     \end{array}
     \right.     
$$
$$
\therefore \vec{v} \cdot \vec{w} = \vec{w} \cdot \vec{v}
$$
\begin{flushright}
$\blacksquare$
\end{flushright}

\textit{Dem. de 3:}
\begin{center}
Sean $\vec{u}, \vec{v}, \vec{w} \in \mathds{R}^n$
\end{center}

$$
    \left.
    \begin{array}{cclc}
     \vec{u} \cdot (\vec{v} + \vec{w}) & = & (u_1, u_2, \cdots, u_n) \cdot [(v_1, v_2, \cdots, v_n) + (w_1, w_2, \cdots, w_n)] & 
  \\  & = & (u_1, u_2, \cdots, u_n) \cdot (v_1 + w_1, v_2 + w_2, \cdots, v_n + w_n)  &
  \\  & = & u_1(v_1 + w_1) + u_2(v_2 + w_2) + u_3(v_3 + w_3) + \cdots + u_n(v_n + w_n) & Def. de <;>
  \\  & = & u_1(v_1) + u_1(w_1) + u_2(v_2) + u_2(w_2) + \cdots +  u_n(v_n) + u_n(w_n) & $Propiedades en $\mathds{R}
  \\  & = & u_1(v_1) + u_2(v_2) + \cdots +  u_n(v_n) + u_1(w_1) + u_2(w_2) + \cdots + u_n(w_n) & Reacomodamos
  \\  & = & (u_1, u_2, \cdots, u_n) \cdot (v_1, v_2, \cdots, v_n) + (u_1, u_2, \cdots, u_n) \cdot (w_1, w_2, \cdots, w_n) & Def. de <;>
  \\  & = & \vec{u} \cdot \vec{v} + \vec{u} \cdot \vec{w}
     \end{array}
     \right.     
$$
$$
\therefore \vec{u} \cdot (\vec{v} + \vec{w}) = \vec{u} \cdot \vec{v} + \vec{u} \cdot \vec{w}
$$
\begin{flushright}
$\blacksquare$
\end{flushright}
Por \'ultimo.
\textit{Dem. de 4:}
\begin{center}
Sean $\vec{v}, \vec{w} \in \mathds{R}^n$ y $\alpha \in \mathds{R}$
\end{center}
$$
    \left.
    \begin{array}{cclc}
     (\alpha \vec{v}) \cdot \vec{w} & = & [\alpha (v_1, v_2, \cdots, v_n)] \cdot (w_1, w_2, \cdots, w_n) & 
  \\  & = & (\alpha v_1, \alpha v_2, \cdots, \alpha v_n) \cdot (w_1, w_2, \cdots, w_n) & Def. $de las operaciones en el $E.V.
  \\  & = & (\alpha v_1)w_1 + (\alpha v_2)w_2 + (\alpha v_3)w_3 + \cdots + (\alpha v_n)w_n & Def. de <;>
  \\  & = & \alpha (v_1 w_1) + \alpha (v_2 w_2) + \alpha (v_3 w_3) + \cdots + \alpha (v_n w_n) & $Propiedades en $\mathds{R}
  \\  & = & \alpha [(v_1, v_2, \cdots, v_n) \cdot (w_1, w_2, \cdots, w_n)] & Def. de <;>
  \\  & = & \alpha (\vec{v} \cdot \vec{w})
     \end{array}
     \right.     
$$
$$
\therefore (\alpha \vec{v}) \cdot \vec{w} = \alpha (\vec{v} \cdot \vec{w})
$$
\begin{flushright}
$\blacksquare$
\end{flushright}




\end{document}