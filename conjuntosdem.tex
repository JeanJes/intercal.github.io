\documentclass[a4paper, 11pt]{article}
\usepackage{comment} % enables the use of multi-line comments (\ifx \fi) 
\usepackage{lipsum} %This package just generates Lorem Ipsum filler text. 
\usepackage{fullpage} % changes the margin
\usepackage[a4paper, total={7in, 10in}]{geometry}
\usepackage[fleqn]{amsmath}
\usepackage{amssymb,amsthm}  % assumes amsmath package installed
\newtheorem{theorem}{Theorem}
\newtheorem{corollary}{Corollary}
\usepackage{graphicx}
\usepackage{tikz}
\usetikzlibrary{arrows}
\usepackage{verbatim}
\usepackage{amsmath} 
\usepackage{float}
\usepackage{xcolor}
\usepackage{mdframed}
\usepackage[shortlabels]{enumitem}
\usepackage{indentfirst}
\usepackage{hyperref}
\usepackage{dsfont}


\begin{document}
\texttt{Propiedad de conjuntos:}\\
Sean A, B conjunos entonces A $\subseteq$ B $\Leftrightarrow$ B$^c$ $\subseteq$ A$^c$\\
\textit{Dem.}\\
$\Rightarrow \rfloor$
\begin{center}
Sea x$\in$B$^c$ por definici\'on de complemento x$\notin$B. \\Por otra parte, si suponemos que  x$\in$A por hip\'otesis entonces x$\in$B, lo que nos lleva a un absurdo pues  x$\notin$B, entonces debe ser que  x$\notin$A de tal manera;\\  si x$\in$ B$^c$ $\Rightarrow$ x$\in$ A$^c$ \\
$\therefore$ B$^c$ $\subseteq$ A$^c$
\end{center}
$\Leftarrow \rfloor$
\begin{center}
Esta parte lo demostraremos por deduccci\'on al absurdo, por lo que supondremos que B$^c$ $\subseteq$ A$^c$ \& A$\nsubseteq$B entonces\\

Sea x$\in$A $\Rightarrow$ x$\in$B\\
Pero si x$\in$B entonces x$\in$B$^c$ y por hip\'otesis x$\in$A$^c$\\
$\therefore$ x$\in$A \& x$\in$A$^c$ \textbf{!}\\
Como suponer que A$\nsubseteq$B nos llev\'o a un absurdo debe ser que \\
A$\subseteq$B
\end{center}
\begin{flushright}
$\blacksquare$
\end{flushright}



















\end{document}